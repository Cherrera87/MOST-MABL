\documentclass[12pt]{article}

%-------------------------------------------------
% Paquetes
%-------------------------------------------------
\usepackage[spanish]{babel}
\usepackage[T1]{fontenc}
\usepackage[utf8]{inputenc}
\usepackage{geometry}
\usepackage{setspace}
\usepackage{amsmath, amssymb}
\usepackage{graphicx}
\usepackage{hyperref}
\usepackage{siunitx}
\usepackage{xcolor}
\usepackage{natbib}

\geometry{margin=1in}
\onehalfspacing

%-------------------------------------------------
% Comandos útiles
%-------------------------------------------------
\newcommand{\ustar}{u_*}
\newcommand{\zetaMO}{z/L}

%-------------------------------------------------
\begin{document}

\begin{center}
{\Large \textbf{Apuntes sobre estabilidad atmosférica en la capa límite marina}}\\[0.2cm]
{\large Teoría de Monin–Obukhov y Eddy Covariance}\\[0.3cm]
\end{center}

\vspace{0.3cm}

%-------------------------------------------------
\section{Introducción}

La capa límite marina (Marine Atmospheric Boundary Layer, MABL) es la región de la atmósfera influenciada directamente por la superficie del océano. En esta capa, los flujos turbulentos de momento (relacionados con el transporte de cantidad de movimiento) y de calor controlan el intercambio aire–mar. También determinan el régimen de estabilidad atmosférica \citep{Garratt1992, Csanady2004}.

En estos apuntes se presenta el marco teórico necesario para interpretar la estabilidad atmosférica en la MABL a partir de mediciones de \emph{eddy covariance}, mediante la teoría de similaridad de Monin–Obukhov (MOST). Esta teoría proporciona una descripción física del equilibrio entre el forzamiento mecánico inducido por el viento y los efectos de flotabilidad asociados a los flujos de calor en la capa superficial de la atmósfera \citep{Monin1954, Kaimal1994}.

El análisis se centra en mediciones realizadas a una altura $z$, medida desde la superficie. Se considera que esta altura se encuentra dentro de la capa superficial de la capa límite marina, un requisito necesario para aplicar la teoría de similaridad de Monin–Obukhov.

%-------------------------------------------------
\section{La capa superficial marina}

Dentro de la MABL, la \emph{capa superficial} es la región más cercana a la superficie (típicamente, los primeros \SIrange{10}{50}{m}), donde los flujos turbulentos de momento y calor pueden considerarse aproximadamente constantes con la altura, siempre que se cumplan las condiciones de homogeneidad horizontal y estacionariedad \citep{Stull1988, Garratt1992}.

En condiciones marinas:
\begin{itemize}
\item la rugosidad superficial está controlada por el oleaje y el viento,
\item los gradientes térmicos suelen ser débiles debido a la alta inercia térmica del océano,
\item Y el régimen casi neutro ocurre con alta frecuencia.
\end{itemize}

Estas características hacen de la MABL un entorno propicio para estudiar la interacción entre el forzamiento mecánico, representado por el flujo de momento, y la flotabilidad asociada a los flujos de calor \citep{Garratt1992, Csanady2004}.

%-------------------------------------------------
\section{Teoría de similaridad de Monin–Obukhov}

La teoría de la similaridad de Monin–Obukhov (MOST) constituye el marco teórico fundamental para describir la estructura turbulenta de la capa superficial atmosférica. Fue formulada originalmente por \citet{Monin1954} a partir de argumentos de análisis dimensional, bajo el supuesto de que, cerca de la superficie, la turbulencia está controlada por un número reducido de variables de escala.

MOST considera que, en la capa superficial:
\begin{itemize}
\item los flujos turbulentos de momento y calor son aproximadamente constantes con la altura,
\item el flujo es horizontalmente homogéneo,
\item y las propiedades estadísticas de la turbulencia dependen únicamente de variables de superficie y de la distancia a la superficie.
\end{itemize}

Bajo estas hipótesis, la dinámica turbulenta puede describirse mediante escalas características asociadas al forzamiento mecánico y a la flotabilidad, sin necesidad de resolver explícitamente el detalle de los remolinos turbulentos \citep{Stull1988, Garratt1992}. La formulación y la validación experimental de las funciones de similaridad para el flujo de momento y el transporte de escalares fueron consolidadas en trabajos clásicos como \citet{Kaimal1994}.

\subsection{Escalas de superficie}

En MOST, la intensidad del forzamiento mecánico se caracteriza mediante la \emph{velocidad de fricción} $\ustar$ (es decir, la velocidad asociada a la transferencia de momento cerca de la superficie), mientras que el papel de la flotabilidad se describe mediante el flujo de calor (preferentemente expresado en términos de temperatura virtual, que considera los efectos de la humedad sobre la densidad del aire). La competencia entre ambos mecanismos se resume en la longitud de Monin–Obukhov $L$ (una escala de longitud que indica el régimen de dominancia entre el forzamiento térmico y el forzamiento mecánico).

\subsection{Longitud de Monin–Obukhov}

La longitud de Monin–Obukhov se define como:
\begin{equation}
L = -\frac{\rho c_p \theta_0 \ustar^3}{\kappa g H}
\end{equation}
Donde $\kappa \approx 0.4$ es la constante de von Kármán, $g$ la aceleración de la gravedad, $\theta_0$ una temperatura de referencia (potencial o virtual) y $H$ el flujo de calor sensible.

El signo de $L$ indica el régimen:
\begin{itemize}
\item $L<0$: condiciones inestables (flotabilidad genera turbulencia),
\item $L>0$: condiciones estables (flotabilidad inhibe turbulencia),
\item $|L|\to \infty$: condiciones cercanas a la neutralidad.
\end{itemize}

%-------------------------------------------------
\section{Flujos turbulentos medidos por Eddy Covariance}

El método de \emph{eddy covariance} permite estimar flujos turbulentos a partir de las covarianzas entre fluctuaciones instantáneas de velocidad y de escalares, proporcionando una vía directa para estimar las escalas de superficie utilizadas por la teoría de Monin–Obukhov \citep{Aubinet2012}.

\subsection{Flujo de momento}

En el caso bidimensional, el esfuerzo cortante turbulento se expresa como:
\begin{equation}
\tau = -\rho \overline{u’w’}
\end{equation}
Donde $u’$ y $w’$ son las fluctuaciones turbulentas de las componentes horizontal y vertical de la velocidad del viento, respectivamente. Esta formulación corresponde a una simplificación común cuando el flujo está alineado con la dirección media del viento \citep{Aubinet2012}.

Se define la velocidad de fricción:
\begin{equation}
\ustar = \sqrt{\frac{|\tau|}{\rho}}
\end{equation}
La magnitud de $\ustar$ representa la intensidad del forzamiento mecánico inducido por el viento.

\subsection{Flujo de calor}

El flujo turbulento de calor sensible se calcula como:
\begin{equation}
H = \rho c_p \overline{w’\theta_v’}
\end{equation}
Donde $\theta_v$ es la temperatura virtual. El signo de $H$ determina el efecto de la flotabilidad:
\begin{itemize}
\item $H>0$: condiciones inestables,
\item $H<0$: condiciones estables.
\end{itemize}

%-------------------------------------------------
\section{El parámetro de estabilidad}

La estabilidad relevante a una altura $z$ se caracteriza mediante:
\begin{equation}
\zetaMO = \frac{z}{L}
\end{equation}
Este parámetro adimensional indica la importancia relativa de la flotabilidad frente al forzamiento mecánico a la altura de medición y constituye la base para clasificar el régimen de estabilidad en la capa superficial \citep{Stull1988, Garratt1992}.

%-------------------------------------------------
\section{Clasificación y régimen casi neutro}

Una clasificación operativa ampliamente utilizada en estudios atmosféricos es \citep{Stull1988, Garratt1992}:
\begin{center}
\begin{tabular}{ll}
\hline
Régimen & Criterio \\ \hline
Inestable & $\zetaMO < -0.1$ \\
Casi neutro & $|\zetaMO| \le 0.1$ \\
Estable & $\zetaMO > 0.1$ \\
\hline
\end{tabular}
\end{center}

Los valores umbral $\pm 0.1$ no representan límites físicos estrictos, sino rangos empíricos derivados de observaciones que indican cuándo los efectos de la flotabilidad comienzan a modificar significativamente la estructura turbulenta con respecto al caso puramente mecánico \citep{Garratt1992}. En particular, el régimen casi neutro no implica un flujo de calor nulo, sino que el efecto térmico es secundario dinámicamente.

Reordenando la definición de $\zetaMO$ se obtiene:
\begin{equation}
\zetaMO =
-\frac{\kappa g z}{\rho c_p \theta_0}
\frac{H}{\ustar^3}
\end{equation}
y la condición $|\zetaMO|\le \zeta_0$ implica:
\begin{equation}
|H| \le
\frac{\rho c_p \theta_0}{\kappa g}
\frac{\zeta_0}{z}
\ustar^3
\end{equation}
Lo cual muestra que la anchura del régimen casi neutro crece con $\ustar^3$ \citep{Garratt1992}.

%-------------------------------------------------
\section{Supuestos y dominio de validez de la teoría de Monin–Obukhov}

La teoría de la similaridad de Monin–Obukhov es una teoría asintótica cuya validez depende del cumplimiento de un conjunto de supuestos físicos bien definidos. Estos supuestos delimitan el dominio en el que los parámetros $\ustar$, $H$, $L$ y $\zetaMO$ pueden interpretarse de manera consistente \citep{Stull1988, Garratt1992}.

\subsection{Aplicabilidad a la capa superficial}

MOST es válida únicamente en la capa superficial atmosférica, entendida como la región más cercana a la superficie, donde los flujos turbulentos de momento y de calor pueden considerarse aproximadamente constantes con la altura. Fuera de esta capa, la influencia de procesos no locales invalida los supuestos básicos de la teoría \citep{Kaimal1994}.

\subsection{Constancia vertical de los flujos}

La teoría asume que los flujos turbulentos de momento y de calor no presentan gradientes verticales significativos en la capa superficial. Esta condición permite que $\ustar$ y $H$ sean tratados como escalas de superficie representativas del estado turbulento local.

\subsection{Homogeneidad horizontal y estacionariedad}

MOST requiere condiciones de homogeneidad horizontal y de estacionariedad estadística. La presencia de fuertes gradientes horizontales, cambios abruptos en la rugosidad o transiciones tierra–mar puede comprometer la validez de la teoría. Asimismo, los flujos deben promediarse en ventanas temporales lo suficientemente largas como para garantizar la convergencia estadística \citep{Aubinet2012}.

\subsection{Turbulencia plenamente desarrollada}

La teoría presupone la existencia de turbulencia plenamente desarrollada. En condiciones muy estables, cuando la turbulencia se vuelve intermitente o queda fuertemente suprimida por la estratificación, la aplicabilidad de MOST se limita.

\subsection{Procesos excluidos}

MOST describe el balance local entre el forzamiento mecánico y la flotabilidad y no incorpora explícitamente otros procesos dinámicos, como la rotación terrestre, la subsidencia a gran escala o los efectos no locales del transporte turbulento. Estos procesos pueden ser relevantes fuera de la capa superficial o en condiciones atmosféricas específicas \citep{Garratt1992}.

En ambientes marinos, la teoría suele ser particularmente útil debido a la mayor homogeneidad superficial; sin embargo, su aplicación debe evaluarse con cautela en presencia de oleaje significativo, desacople aire–mar o estratificación térmica intensa \citep{Csanady2004}.

%-------------------------------------------------
\begin{table}[h!]
\centering
\caption{Supuestos fundamentales de la teoría de la similaridad de Monin–Obukhov (MOST), sus implicaciones físicas y sus principales limitaciones en la capa límite marina.}
\label{tab:supuestos_MOST}
\renewcommand{\arraystretch}{1.3}
\begin{tabular}{p{3.5cm} p{6.5cm} p{4.5cm}}
\hline
\textbf{Supuesto} & \textbf{Implicación física} & \textbf{Cuándo puede fallar} \\
\hline
Aplicabilidad a la capa superficial &
MOST describe únicamente la región más cercana a la superficie, donde los flujos turbulentos de momento y de calor pueden considerarse constantes con la altura. &
Fuera de la capa superficial: presencia de efectos no locales o de acoplamiento con las capas superiores de la MABL. \\
Constancia vertical de los flujos &
Permite definir escalas de superficie únicas ($\ustar$, $H$) que representan el estado turbulento local. &
En presencia de fuertes gradientes verticales de flujo, de cizallamiento inducido por chorros de bajo nivel o de estratificación intensa. \\
Homogeneidad horizontal &
El campo turbulento depende solo de la distancia vertical a la superficie, no de la posición horizontal. &
Zonas costeras, transiciones tierra–mar, frentes térmicos o cambios abruptos en la rugosidad. \\
Estacionariedad estadística &
Las covarianzas turbulentas representan promedios temporales estables, necesarios para definir flujos significativos. &
Condiciones transitorias, cambios rápidos de viento o de estabilidad o ventanas de promediado insuficientes. \\
Turbulencia plenamente desarrollada &
La producción de turbulencia por corte y/o flotabilidad domina sobre los procesos disipativos o intermitentes. &
Condiciones muy estables, especialmente nocturnas, con turbulencia intermitente o colapsada. \\
Dominancia de corte mecánico y flotabilidad &
El equilibrio local entre los forzamientos mecánicos y térmicos determina la estructura turbulenta. &
Presencia de procesos no incluidos en MOST: rotación, subsidencia, transporte no local, ondas internas. \\
\hline
\end{tabular}
\end{table}

%-------------------------------------------------
\bibliographystyle{plainnat}
\bibliography{referencias_MOST}

\end{document}