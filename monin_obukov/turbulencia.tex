\documentclass[12pt]{book}

%-------------------------------------------------
% Paquetes
%-------------------------------------------------
\usepackage[spanish]{babel}
\usepackage[T1]{fontenc}
\usepackage[utf8]{inputenc}
\usepackage{geometry}
\usepackage{setspace}
\usepackage{amsmath, amssymb}
\usepackage{graphicx}
\usepackage{siunitx}
\usepackage{natbib}

\geometry{margin=1in}
\onehalfspacing

%-------------------------------------------------
% Comandos
%-------------------------------------------------
\newcommand{\ustar}{u_*}

%-------------------------------------------------
\begin{document}
%=================================================
% SECCIÓN PARA AÑADIR: de Navier--Stokes a esfuerzos de Reynolds (RANS)
% y derivación de la ecuación de energía cinética turbulenta (TKE)
%=================================================

\chapter{Turbulencia -- Esfuerzos de Reynolds}
\label{sec:RANS_Reynolds}

En esta sección se muestra cómo surgen los esfuerzos de Reynolds al promediar la ecuación de momento y por qué el método de \emph{eddy covariance} permite estimarlos directamente a partir de la covarianza de las fluctuaciones turbulentas \citep{Stull1988, Garratt1992, Kaimal1994, Aubinet2012}.

\subsection{Ecuación instantánea de momento}
\label{subsec:NS}

Partimos de la ecuación de Navier--Stokes en notación indicial:
\begin{equation}
\frac{\partial u_i}{\partial t} + u_j \frac{\partial u_i}{\partial x_j}
= -\frac{1}{\rho}\frac{\partial p}{\partial x_i}
+ \nu \frac{\partial^2 u_i}{\partial x_j^2} + f_i,
\label{eq:NS}
\end{equation}
Aquí, $u_i$ es la velocidad instantánea, $p$ la presión, $\rho$ la densidad, $\nu$ la viscosidad cinemática y $f_i$ representa las fuerzas de cuerpo (o fuerzas volumétricas por unidad de masa), tales como la gravedad o la fuerza de Coriolis.

Luego de presentar la ecuación instantánea, examinamos el proceso de promediado y la descomposición de Reynolds.

Se define un promedio de Reynolds (temporal o de ensamble) $\overline{(\cdot)}$ y se introduce la descomposición:
\begin{equation}
u_i = \overline{u_i} + u_i',
\qquad
p = \overline{p} + p',
\qquad
\overline{u_i'}=0,
\qquad
\overline{p'}=0.
\label{eq:Reynolds_decomp}
\end{equation}
Se considera que el operador de promedio es lineal y que conmuta (aproximadamente) con las derivadas espacio--temporales bajo condiciones de estacionariedad/homogeneidad en el intervalo promediado \citep{Stull1988, Garratt1992}.

Continuando con el análisis de la ecuación de momento, se estudian la importancia del término no lineal y el origen de los esfuerzos de Reynolds.

A partir del término advectivo \(u_j \partial_{x_j}u_i\) de la ecuación \eqref{eq:NS}, se realiza la descomposición de Reynolds \eqref{eq:Reynolds_decomp}, tal que:
\begin{align}
u_j\frac{\partial u_i}{\partial x_j}
&= (\overline{u_j}+u_j')\frac{\partial (\overline{u_i}+u_i')}{\partial x_j}
\nonumber\\
&= \overline{u_j}\frac{\partial \overline{u_i}}{\partial x_j}
+ \overline{u_j}\frac{\partial u_i'}{\partial x_j}
+ u_j'\frac{\partial \overline{u_i}}{\partial x_j}
+ u_j'\frac{\partial u_i'}{\partial x_j}.
\label{eq:expand_convective}
\end{align}
Promediando término a término y usando $\overline{u_j'}=0$ y $\overline{u_i'}=0$:
\begin{equation}
\overline{u_j\frac{\partial u_i}{\partial x_j}}
=
\overline{u_j}\frac{\partial \overline{u_i}}{\partial x_j}
+
\overline{u_j'\frac{\partial u_i'}{\partial x_j}}.
\label{eq:avg_convective_1}
\end{equation}

Para reescribir el último término, usamos la identidad de producto:
\begin{equation}
\frac{\partial (u_i' u_j')}{\partial x_j}
=
u_j'\frac{\partial u_i'}{\partial x_j}
+
u_i'\frac{\partial u_j'}{\partial x_j}.
\label{eq:product_rule}
\end{equation}

De aquí:
\begin{equation}
u_j'\frac{\partial u_i'}{\partial x_j}
=
\frac{\partial (u_i' u_j')}{\partial x_j}
-
u_i'\frac{\partial u_j'}{\partial x_j}.
\label{eq:rearrange}
\end{equation}

Promediando:
\begin{equation}
\overline{u_j'\frac{\partial u_i'}{\partial x_j}}
=
\frac{\partial \overline{u_i' u_j'}}{\partial x_j}
-
\overline{u_i'\frac{\partial u_j'}{\partial x_j}}.
\label{eq:avg_convective_intermediate}
\end{equation}

Bajo la condición de incompresibilidad (o aproximación de Boussinesq en la capa superficial), consideramos
\begin{equation}
\frac{\partial u_j'}{\partial x_j} \approx 0,
\end{equation}
de modo que
\begin{equation}
\overline{u_i'\frac{\partial u_j'}{\partial x_j}} \approx 0,
\end{equation}
y por tanto:
\begin{equation}
\overline{u_j'\frac{\partial u_i'}{\partial x_j}}
=
\frac{\partial \overline{u_i' u_j'}}{\partial x_j}.
\label{eq:avg_convective_2}
\end{equation}

Sustituyendo \eqref{eq:avg_convective_2} en \eqref{eq:avg_convective_1}:
\begin{equation}
\overline{u_j\frac{\partial u_i}{\partial x_j}}
=
\overline{u_j}\frac{\partial \overline{u_i}}{\partial x_j}
+
\frac{\partial \overline{u_i' u_j'}}{\partial x_j}.
\label{eq:avg_convective_final}
\end{equation}
El término $\overline{u_i'u_j'}$  es el tensor de las correlaciones turbulentas, que representa el transporte adicional de momento por turbulencia.

Una vez identificado el origen de los esfuerzos de Reynolds, presentamos la ecuación de momento promediada (RANS) y definimos el tensor de esfuerzos de Reynolds.

Promediando \eqref{eq:NS} y usando \eqref{eq:avg_convective_final}:
\begin{equation}
\frac{\partial \overline{u_i}}{\partial t}
+
\overline{u_j}\frac{\partial \overline{u_i}}{\partial x_j}
=
-\frac{1}{\rho}\frac{\partial \overline{p}}{\partial x_i}
+ \nu \frac{\partial^2 \overline{u_i}}{\partial x_j^2}
-\frac{\partial \overline{u_i' u_j'}}{\partial x_j}
+ \overline{f_i}.
\label{eq:RANS}
\end{equation}
Definimos el tensor de esfuerzos de Reynolds:
\begin{equation}
\tau^{(t)}_{ij} \equiv -\rho\,\overline{u_i' u_j'}.
\label{eq:Reynolds_stress_tensor}
\end{equation}
En la capa superficial de la atmósfera, el término más relevante para el intercambio vertical de momento es el esfuerzo cortante turbulento:
\begin{equation}
\tau \equiv \tau^{(t)}_{xz} = -\rho\,\overline{u'w'},
\qquad
(\text{y en 3D } \; \boldsymbol{\tau} = -\rho(\overline{u'w'},\overline{v'w'})).
\label{eq:shear_stress}
\end{equation}
Esto explica por qué el método de  \emph{eddy covariance} permite estimar directamente el transporte turbulento de momento: se mide $\overline{u'w'}$ (y $\overline{v'w'}$) a partir de series de alta frecuencia, obteniendo así el esfuerzo de Reynolds \citep{Kaimal1994, Aubinet2012}.

%=================================================
Con las bases anteriores, ahora derivamos la ecuación de la energía cinética turbulenta (TKE).

La energía cinética turbulenta se define como:
\begin{equation}
e \equiv \frac{1}{2}\overline{u_i' u_i'}.
\label{eq:TKE_def}
\end{equation}
La ecuación de energía cinética turbulenta (TKE) se obtiene multiplicando la ecuación de momento de las fluctuaciones por $u_i'$ y promediando \citep{Stull1988, Garratt1992, Kaimal1994}.

\subsection{Ecuación de momento para las fluctuaciones}
\label{subsec:fluct_momentum}

Restando la ecuación promediada \eqref{eq:RANS} de la ecuación instantánea \eqref{eq:NS}, se obtiene la ecuación para $u_i'$ (aquí se muestra una forma estándar bajo Boussinesq e incompresibilidad para enfatizar los términos turbulentos):
\begin{equation}
\frac{\partial u_i'}{\partial t}
+ \overline{u_j}\frac{\partial u_i'}{\partial x_j}
+ u_j'\frac{\partial \overline{u_i}}{\partial x_j}
+ \left(u_j'\frac{\partial u_i'}{\partial x_j} - \overline{u_j'\frac{\partial u_i'}{\partial x_j}}\right)
=
-\frac{1}{\rho}\frac{\partial p'}{\partial x_i}
+ \nu \frac{\partial^2 u_i'}{\partial x_j^2}
+ b_i',
\label{eq:fluct_mom}
\end{equation}
Aquí, $b_i'$ es la contribución fluctuante de flotabilidad (por ejemplo, en vertical, $b_3' \approx g\,\theta_v'/\theta_0$ en Boussinesq). Su forma exacta depende de la formulación termodinámica adoptada \citep{Stull1988, Garratt1992}.

Como siguiente paso, multiplicamos \eqref{eq:fluct_mom} por $u_i'$ y promediamos.

Multiplicamos \eqref{eq:fluct_mom} por $u_i'$ y promediamos y utilizando las identidades:
\begin{equation}
u_i'\frac{\partial u_i'}{\partial t} = \frac{\partial}{\partial t}\left(\frac{u_i'u_i'}{2}\right),
\qquad
\overline{u_i' \overline{u_j}\frac{\partial u_i'}{\partial x_j}}
=
\overline{u_j}\frac{\partial}{\partial x_j}\left(\frac{\overline{u_i'u_i'}}{2}\right),
\label{eq:identities_time_adv}
\end{equation}

Tras reorganizar, se obtiene la ecuación de TKE:

\subsection{Ecuación de energía cinética turbulenta (TKE)}
\label{subsec:TKE_budget}

\begin{align}
\frac{\partial e}{\partial t}
+ \overline{u_j}\frac{\partial e}{\partial x_j}
&=
\underbrace{-\overline{u_i'u_j'}\frac{\partial \overline{u_i}}{\partial x_j}}_{\text{Producción por corte }(P)}
\;+\;
\underbrace{\overline{b_i' u_i'}}_{\text{Producción/consumo por flotabilidad }(B)}
\;-\;
\underbrace{\frac{\partial}{\partial x_j}\left(\frac{1}{2}\overline{u_i'u_i'u_j'}\right)}_{\text{Transporte turbulento}}
\nonumber\\
&\quad
-\underbrace{\frac{1}{\rho}\frac{\partial \overline{p' u_j'}}{\partial x_j}}_{\text{Transporte por presión}}
+\underbrace{\nu\frac{\partial^2 e}{\partial x_j^2}}_{\text{Transporte viscoso}}
-\underbrace{\nu\,\overline{\frac{\partial u_i'}{\partial x_j}\frac{\partial u_i'}{\partial x_j}}}_{\text{Disipación }(\varepsilon)}.
\label{eq:TKE_budget}
\end{align}

En la meteorología de capa límite, el término de flotabilidad suele escribirse explícitamente como:
\begin{equation}
B \;=\; \frac{g}{\theta_0}\,\overline{w'\theta_v'}
\qquad
(\text{Boussinesq; usando temperatura virtual}).
\label{eq:buoyancy_term}
\end{equation}
El término de producción por corte es el puente directo hacia la teoría de similaridad de Monin--Ovukov (MOST), ya que en capa superficial típicamente domina la contribución vertical:
\begin{equation}
P \approx -\overline{u'w'}\,\frac{\partial \overline{u}}{\partial z}
\quad
(\text{y análogo para } v).
\label{eq:shear_production_surface}
\end{equation}
La importancia relativa entre la producción por corte ($P$) y la producción por flotabilidad ($B$) es precisamente la motivación física detrás de la longitud de Monin--Obukhov y del parámetro de estabilidad $z/L$ \citep{Stull1988, Garratt1992, Kaimal1994}.

\subsection{Reducciones típicas en la capa superficial}
\label{subsec:TKE_surface_layer}

Bajo condiciones cercanas a la homogeneidad horizontal y la estacionariedad (aproximación de capa superficial), la ecuación de TKE \eqref{eq:TKE_budget} se simplifica a una forma 1D vertical:
\begin{equation}
0 \approx
-\overline{u'w'}\,\frac{\partial \overline{u}}{\partial z}
+\frac{g}{\theta_0}\overline{w'\theta_v'}
-\frac{\partial}{\partial z}\left(
\frac{1}{2}\overline{u_i'u_i'w'}
+\frac{1}{\rho}\overline{p'w'}
-\nu\frac{\partial e}{\partial z}
\right)
-\varepsilon,
\label{eq:TKE_1D}
\end{equation}
Lo que enfatiza que, aun en la capa superficial, el balance local entre producción (corte y flotabilidad), transporte vertical y disipación depende del régimen de estabilidad \citep{Kaimal1994, Garratt1992}.


\bibliographystyle{plainnat}
\bibliography{referencias_MOST}

\end{document}
